\section{Analisi dei risultati}
\label{cap:performance-analysis}

\subsection{Domanda \#1}
\label{sec:question-1}

\begin{displayquote}
Misurate i tempi di calcolo della procedura
\codeinline{full\_contraction} sui grafi del dataset. Mostrate i
risultati con un grafico che mostri la variazione dei tempi di calcolo
al variare del numero di vertici nel grafo. Confrontate i tempi
misurati con la complessità asintotica di
\codeinline{full\_contraction}.
\end{displayquote}

\noindent Il grafico \ref{fig:karger-full-contraction-chart} mostra la
variazione dei tempi di calcolo al variare del numero dei vertici del
grafo ed evidenzia la differenza del tempo di esecuzione pratico di
\codeinline{full\_contraction} rispetto a quello asintotico. In
realtà, come è possibile vedere, a meno di un certo fattore, anche se
abbastanza significativo, le due curve hanno la stessa complessità
asintotica.

\begin{figure}[H]
    \centering

    \includegraphics[width=0.9\textwidth]{./images/Tempo_di_esecuzione_di_full_contraction_rispetto_al_numero_di_nodi.png}

    \caption{Confronto tra il tempo di esecuzione di \codeinline{full\_contraction} e la sua complessità asintotica rispetto al numero di nodi. Grafico in scala logaritmica.}
    \label{fig:karger-full-contraction-chart}
\end{figure}

\subsection{Domanda \#2}
\label{sec:question-2}

\begin{displayquote}
Misurate i tempi di calcolo dell'algoritmo di Karger sui grafi del dataset, usando un numero di ripetizioni che garantisca una probabilità minore o uguale a $\frac{1}{n}$ di sbagliare. Mostrate i risultati con un grafico che mostri la variazione dei tempi di calcolo al variare del numero di vertici nel grafo. Confrontate i tempi misurati con la complessità asintotica dell'algoritmo. \\


Per permettere all'algoritmo di karger un probabilità di fallimento pari a $\frac{1}{n}$ è necessario settare d = 1 e di conseguenza ricavare k con la formula vista nella sezione \ref{sub:karger-success-probability} e quindi:
$$ k = 1 * \frac{n^2}{2} * ln(n)$$

Nelle istanze più grandi, il tempo di calcolo necessario a completare tutte le iterazioni potrebbe risultare eccessivo. In questo caso utilizzate un timeout oppure abbassate il numero di ripetizioni per ottenere tempi di esecuzione ragionevoli.
\end{displayquote}

\subsection{Domanda \#3}
\label{sec:question-3}

\begin{displayquote}
Misurate il \textit{discovery time} dell'algoritmo di Karger sui grafi del dataset. Il discovery time è il momento (in secondi) in cui l'algoritmo trova per la prima volta il taglio di costo mimimo.  Confrontate il discovery time con il tempo di esecuzione complessivo per ognuno dei grafi nel dataset.
\end{displayquote}

\subsection{Domanda \#4}
\label{sec:question-4}

\begin{displayquote}
Per ognuno dei grafi del dataset, riportate il risultato risultato ottenuto dalla vostra implementazione, la soluzione attesa e l'errore relativo calcolato come:

\begin{equation*}
    \frac{SoluzioneTrovata - SoluzioneAttesa}{SoluzioneAttesa}
\end{equation*}

\end{displayquote}
