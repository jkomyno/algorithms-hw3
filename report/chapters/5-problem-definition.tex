\section{Definizione del problema minimum-cut}
\label{cap:problem-definition}

Sia $ G = (V,E) $ un grafo non diretto con $ n $ vertici e $ m $ lati. Siamo interessati alla nozione di \textit{taglio} di un grafo.

\begin{defn}
Un \textbf{taglio} (cut) in $ G $ è un partizionamento dei vertici $ V $ in 2 insiemi, $ S $ and $ T $, $ T=V(G) \setminus S$, dove gli archi di un taglio sono $(S, T) = \{ uv \shortmid u \in S, v \in T, S \cap T = V, uv \in E \}$,
dove $ S \neq \emptyset $ e $ T \neq \emptyset $. Il numero di archi nel taglio $ (S,T) $ è noto come \textit{dimensione} del taglio.
\end{defn}

\noindent Siamo interessati nel problema del calcolo del \textit{minimum cut}, ovvero il taglio di cardinalità minima nel grafo.
Formalmente, vogliamo trovare $ S\subseteq V $ tale che $ (S, T) $ sia minimo, $ S \neq \varnothing $ e $ S \neq T $.
Nella letteratura, il problema è formalizzato in due modi:

\begin{itemize}
    \item nel \textit{s-t min-cut problem}, due vertici particolari $s$ e $t$ sono richiesti essere nelle due parti opposte del taglio;
    \item nel \textit{global min-cut problem}, non si fa riferimento ad alcun vertice specifico.
\end{itemize}

\noindent In questo homework tratteremo solo del \textit{global min-cut problem}. \\
\noindent Questo problema ha numerose applicazioni, ad esempio:

\begin{itemize}
    \item determinare la solidità del design di una rete di calcolatori;
    \item nell'ambito dell'Information Retrieval, identificare cluster di documenti correlati da certi topic in comune;
    \item nella progettazione di compilatori per linguaggi paralleli;
    \item nell'ottimizzazione combinatoria su larga scala;
    \item e molte altre applicazioni \ldots
\end{itemize}
