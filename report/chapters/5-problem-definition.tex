\section{Definizione del problema minimum-cut}
\label{cap:problem-definition}

% Nella teoria dei grafi, un \emph{minimum cut} di $G$ è un taglio
% minimo rispetto a quelche nozione di distanza tra tagli.




\begin{defn}{\emph{(Taglio)}}
  Dato $G = (V,E)$ non diretto, connesso, con $n$ vertici e $m$ lati;
  un taglio (cut) $C \subseteq E$ è un insieme di lati tali che $G' =
  (V, E \setminus C)$ non è connesso.
\end{defn}

\noindent Un taglio è dunque una partizione dei vertici di un grafo in due insiemi disgiunti. Dalla definizione è immediato osservare che $G'$ deve avere
almeno due componenti connesse; nel nostro caso ci interessa
ottenere esattamente due componenti connesse. \\

\noindent Siamo interessati al calcolo del \textit{minimum cut}, che nel caso dei grafi non diretti e non pesati è il taglio di cardinalità minima nel grafo. Formalmente, il minimum cut è un taglio $C \subseteq E$ che minimizza $\abs{C}$.\\

\noindent In letteratura, il problema è formalizzato in due modi:

\begin{itemize}
    \item \textit{s-t min-cut problem}, due vertici particolari $s$ e
      $t$ sono richiesti essere nelle due parti opposte del taglio;
    \item \textit{global min-cut problem}, non si fa riferimento ad
      alcun vertice specifico.
\end{itemize}

\noindent In questo homework affronteremo solo il \textit{global
  min-cut problem}. Questo tipo di problema ha numerose applicazioni, come ad esempio:

\begin{itemize}
    \item determinare la solidità del design di una rete di
      calcolatori;
    \item nell'ambito dell'Information Retrieval, identificare cluster
      di documenti correlati da certi topic in comune;
    \item nella progettazione di compilatori per linguaggi paralleli;
    \item nell'ottimizzazione combinatoria su larga scala.
\end{itemize}
