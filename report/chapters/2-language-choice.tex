\section{IDE e compilatore}
\label{cap:language-choice}

Poiché il nostro sistema operativo di sviluppo è Windows 10, abbiamo usato l'IDE Visual Studio 2019 Community e il suo compilatore \codeinline{MSVC v142 x64/x86}. \\

\noindent Nell'archivio allegato a questa relazione abbiamo incluso un \codeinline{Makefile} per permettere la compilazione su altri sistemi operativi usando \codeinline{g++-9}. Il comando da usare per la compilazione è \mintinline{bash}{make all}. Nel caso la \textit{major version} installata di \codeinline{g++} sia la 9 ma l'alias esplicito \codeinline{g++-9} non esista, è possibile sovrascrivere il compilatore usato con il comando \mintinline{bash}{make CXX=g++ all}. \\

\noindent Altri comandi sono disponibili per eseguire il benchmark degli algoritmi, oppure per eseguire semplicemente i programmi compilati. Ci si riferisca al file \codeinline{README.md} incluso al progetto.
