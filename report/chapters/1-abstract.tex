\section{Abstract}
\label{cap:abstract}

Questo terzo homework di laboratorio di Algoritmi Avanzati ha lo scopo di implementare e valutare l'algoritmo randomizzato di Karger per il problema del minimum cut di un grafo. I grafi considerati sono connessi, non pesati, non diretti e privi di \textit{self-loop}. \\

\noindent I parametri da considerare sono:

\begin{enumerate}
    \item Il tempo impiegato dalla procedura \codeinline{full\_contraction};
    \item Il tempo impiegato dall'algoritmo completo per ripetere la contrazione un numero sufficientemente alto di volte;
    \item Il \textit{discovery time}, ovvero il tempo necessario all'algoritmo ad individuare il taglio di costo minimo la prima volta;
    \item L'errore nella soluzione trovata rispetto al risultato ottimo.
\end{enumerate}

\noindent Abbiamo considerato anche alcuni contributi originali rispetto agli algoritmi visti in classe; esse sono discussi e presentati nella sezione \hyperref[cap:extensions-and-originalities]{Estensioni e originalità}. \\

\noindent Il codice è scritto in C++17 ed è opportunamente commentato per facilitarne la comprensione. Non è stata usata alcuna libreria esterna.

\noindent Le risposte alle 4 domande principali dell'homework sono riportate nella sezione \hyperref[cap:performance-analysis]{Analisi dei risultati}.
