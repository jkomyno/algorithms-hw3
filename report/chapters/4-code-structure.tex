\section{Struttura del codice}
\label{cap:code-structure}

Il progetto è strutturato in un'unica soluzione Visual Studio\footnote{Una soluzione Visual Studio può essere vista come un macro-progetto che contiene più sotto-moduli.} contenente molteplici progetti, uno per ogni algoritmo per il minimum-cut implementato. Il codice di ogni progetto è contenuto nell'omonima cartella. Di seguito l'elenco dei progetti realizzati:

\begin{itemize}
    \item \textbf{KargerMinCut}: Implementazione dell'algoritmo randomizzato di Karger visto a lezione (1994);
    \item  $(*)$ \textbf{KargerSteinMinCut}: Implementazione dell'algoritmo di Karger \& Stein (1997), che migliora di un ordine di grandezza le performance del semplice algoritmo di Karger.
\end{itemize}

\noindent I progetti indicati con $(*)$ sono delle estensioni o delle aggiunte rispetto all'algoritmo inizialmente richiesto dall'homework.
\\

\noindent La cartella \textit{Shared} contiene le strutture dati custom e alcune classi e metodi di utilità usati
condivisi tra progetti. Abbiamo configurato Visual Studio per importare automaticamente i file di header salvati nella cartella \textit{Shared}
durante la compilazione di ogni sottoprogetto. Analogamente, tale cartella è inclusa nella compilazione dal \textit{Makefile}, grazie all'opzione \textit{-I} del compilatore \textit{g++}.

