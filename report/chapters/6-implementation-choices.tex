\section{Scelte implementative}
\label{cap:implementation-choices}

\subsection{Rappresentazione del grafo}
\label{sub:graph-representation}

Gli algoritmi di questo homework operano su (multi)grafi non pesati,
connessi e non diretti.\\

% \noindent Come nel precedente homework, per semplificare la logica di indicizzazione dei nodi del grafo, la label dei nodi (originariamente numerata da $1$ a $n$) è decrementata di 1, quindi i nodi sono rappresentati dall'intervallo numerico $[0, n-1]$.

\noindent La classe che rappresenta la Mappa di Adiacenza dei grafi è
definita in \codeinline{AdjacencyMapGraph.h} nella cartella
\textit{Shared}. Abbiamo modificato la classe utilizzata nel primo homework per supportare le funzionalità richieste da questo nuovo problema. Tali estensioni sono descritte di seguito.

\paragraph{Multigrafo}
\codeinline{AdjacencyMapGraph} è stata corredata da un'ulteriore
campo \codeinline{edge\_count\_map} che permette, dato un arco $(u,v)$, di
accedere al numero di archi $(u,v)$ presenti nel multigrafo. Il tipo di tale attributo è una mappa chiave-valore \codeinline{std::unordered\_map}.

\noindent La chiave della mappa ha tipo \codeinline{Edge}, che rappresenta un generico arco non pesato e non diretto $(u,v)$. Abbiamo definito l'operatore di uguaglianza di \codeinline{Edge} per ottenere la commutatività, ovvero considerare l'arco $(u, v)$ identico all'arco $(v, u)$, ai fini di avere consistenza nel tracciamento degli archi del multigrafo. Anche la funzione hash definita per il tipo \codeinline{Edge}, riportata nel listato  \ref{listing:hash-fun}, è commutativa.
\noindent Il valore della mappa è di tipo intero senza segno \codeinline{size\_t} e rappresenta la cardinalità dell'insieme di archi $(u,v)$ corrispondenti alla chiave. \\

\noindent \codeinline{edge\_count\_map} è essenziale nel tracciamento degli archi durante la contrazione del grafo descritta dalla procedura  \codeinline{full\_contration} nella sezione \ref{sub:karger-definition}.\\

\begin{listing}[!ht]
\begin{minted}{c++}
// funzione di hash commutativa per Edge
struct edge_hash {
    std::size_t operator()(const Edge& edge) const noexcept {
        constexpr auto hash_max = std::numeric_limits<size_t>::max();
        const auto& [i, j] = edge;
        return (i * j + (i * i) * (j * j) + (i * i * i) * (j * j * j)) % hash_max;
    }
};
\end{minted}
\caption{Funzione di hash commutativa per la chiave di tipo Edge della \textit{std::unordered\_map}.}
\label{listing:hash-fun}
\end{listing}

\paragraph{Contrazione di due nodi}

Il nuovo metodo \codeinline{AdjacencyMapGraph::contract} effettua la contrazione di due vertici, chiamati \codeinline{contracted} e \codeinline{incorporator}. \codeinline{contracted} indica il nodo da contrarre, mentre \codeinline{incorporator} è il vertice che deve acquisire gli archi incidenti al nodo contratto. \\

\noindent Gli step eseguiti dalla procedura \codeinline{contract}, la cui implementazione è riportata nel listato \ref{listing:met-contract}, sono i seguenti:

\begin{enumerate}
    \item Eliminare tutti gli archi presenti tra
      \codeinline{contracted} e \codeinline{incorportator} dalla Mappa di Adiacenza e da \codeinline{edge\_count\_map}.

    \item Trasferire ogni arco incidente in
      \textit{contracted} su \textit{incorportator}, aggiornando anche il numero di archi corrispondenti in \codeinline{edge\_count\_map}.

    \item Eliminare dalla Mappa di Adiacenza tutti i riferimenti agli archi incidenti al vertice \codeinline{contracted} e il vertice stesso.
\end{enumerate}

\begin{listing}[!ht]
\begin{minted}{c++}
void AdjacencyMapGraph::contract(size_t contracted, size_t incorporator) {
    // Step 1
    edge_count_map.erase({contracted, incorporator});
    adj_map[incorporator].erase(contracted);
    adj_map[contracted].erase(incorporator);

    // Step 2
    for (const auto node : adj_map.at(contracted)) {
        adj_map[node].erase(contracted);
        adj_map[node].insert(incorporator);
        adj_map[incorporator].insert(node);

        const size_t n_multi_edge = edge_count_map.at({contracted, node});
        edge_count_map.erase({contracted, node});
        edge_count_map[{incorporator, node}] += n_multi_edge;
    }

    // Step 3
    adj_map.erase(contracted);
}
\end{minted}
\caption{Implementazione del metodo \codeinline{contract} di \codeinline{AdjancencyMapGraph}.}
\label{listing:met-contract}
\end{listing}

Si noti che il multigrafo risultante dalla contrazione è sempre privo di \textit{self-loop}.

\paragraph{Selezione di arco in modo casuale}

\codeinline{AdjacencyMapGraph::get\_random\_edge} è un ulteriore metodo aggiunto alla Matrice di Adiacenza, ed è utile a scegliere casualmente due vertici distinti da contrarre. Esso estrae in maniera casuale un arco da \codeinline{edge\_count\_map}.
Non esiste alcuna soluzione \textit{out-of-the-box} per estrarre casualmente elementi da una struttura dati non ordinata come \codeinline{std::unordered\_map}. Abbiamo pensato a due possibili approcci. \\

\noindent L'approccio più semplice consiste nel ritornare sempre l'elemento corrispondente all'iteratore \codeinline{edge\_count\_map.begin()}, sfruttando il fatto che la struttura è non ordinata. Quest'approccio ha tempo di esecuzione costante.
Ci siamo però accorti sperimentalmente che \codeinline{C++} usa al suo interno una procedura deterministica per allocare elementi in \codeinline{std::unordered\_map}. Ogni esecuzione di \codeinline{full\_contraction} risulterebbe quindi identica alle altre, perché verrebbe estratta sempre la stessa sequenza di vertici da contrarre, impedendo all'algoritmo di Karger di trovare il \textit{minimum cut} con alta probabilità. Abbiamo quindi scartato quest'idea. \\

\noindent L'approccio che abbiamo scelto consiste invece nel fare avanzare di una posizione uniformemente casuale l'iteratore di \codeinline{edge\_count\_map}, restituendo l'arco corrispondente. La struttura dati non è modificata da questo processo. Questo approccio è corretto perché garantisce che tutte le selezioni senza ripetizione di coppie di vertici da contrarre siano indipendenti. Nel caso peggiore, quest'operazione è lineare nel numero di archi (senza ripetizione) del grafo. \\

\noindent L'implementazione di \codeinline{get\_random\_edge} è riportata nel listato \ref{listing:get-random-edge}.

\begin{listing}[!ht]
\begin{minted}{c++}
const Edge AdjacencyMapGraph::get_random_edge() const noexcept {
    using namespace random_generator;
    IntegerRandomGenerator random_edge(0, edge_count_map.size() - 1);

    // estrae un arco a caso da edge_count_map
    const auto& [edge, _] = *(std::next(edge_count_map.cbegin(), random_edge()));

    return edge;
}
\end{minted}
\caption{Implementazione del metodo \codeinline{get\_random\_edge} di \codeinline{AdjancencyMapGraph}.}
\label{listing:get-random-edge}
\end{listing}

\subsection{Lettura del Grafo}

\noindent Il file \codeinline{main.cpp} ha la stessa struttura per
ogni algoritmo implementato, come riportato nel listato \ref{listing:main-cpp}.
L'input di ogni programma è un file contentente una lista di
adiacenza, dove, per ogni riga:

\begin{itemize}
    \item la prima colonna indica la label del vertice $u$;
    \item gli elementi successivi formano la lista di tutti i vertici
      incidenti a $u$, cioè i vertici $v$ tali che \\ $\exists$ $(u, v) \in E$.
\end{itemize}

\noindent Ad alto livello, le operazioni svolte da ogni programma sono:

\begin{enumerate}
    \item Lettura dell'input: il file di input viene processato
      da \codeinline{read\_file.h}. Viene letta una riga per volta in
      un buffer, e il primo elemento di tale buffer è usato per
      etichettare il nodo della lista di adiacenza letta. Abbiamo usato la libreria di file streaming nativa di
    C++ (\codeinline{fstream}).

    \item Una volta letti i nodi, viene creata la Mappa di Adiancenza
      nella memoria heap, e ne viene ritornato uno
      \textit{smart-pointer} di tipo
      \mintinline{c++}{std::shared_ptr}.
\end{enumerate}

\begin{listing}[!ht]
\begin{minted}{c++}
int main(int argc, char** argv) {
    if (argc != 2) {
        std::cerr << "1 argument required: filename" << std::endl;
        exit(1);
    }

    const char* filename = argv[1];

    // inizia a misurare il tempo di esecuzione
    const auto program_time_start = stopwatch::now();

    // legge il grafo dal file di input
    auto graph = read_file(filename);

    // numero di iterazioni richieste stimato
    const size_t k = // ...
    std::cout << "k: "s << k << '\n';

    // calcola il min-cut approssimato
    const auto min_cut = // ...

    // ferma l'orologio
    auto program_time_stop = stopwatch::now();

    // calcola il tempo totale di esecuzione in microsecondi
    const auto program_time =
        stopwatch::duration<stopwatch::us_t>(program_time_start, program_time_stop);

    // stampa la soluzione e le statistiche di esecuzione
    std::cout << "min_cut: "s << min_cut << std::endl;
    std::cout << "program_time: "s << program_time << std::endl;
    std::cout << /* ... */ << std::endl;
}
\end{minted}
\caption{Scheletro comune ad ogni file \codeinline{main.cpp} del progetto.}
\label{listing:main-cpp}
\end{listing}

\noindent Tutti i file citati precedentemento sono locati nella
cartella \textit{Shared} del progetto consegnato e sono corredati di
ulteriori commenti esplicativi.

%\subsection{Strutture Dati comuni}

%Tutte le strutture dati elencate di seguito sono definite nella cartella \textit{Shared}.
%Ove possibile, per la nomenclatura dei metodi abbiamo cercato di seguire lo stesso standard dei container STL di C++.
%Inoltre, le strutture dati usate sono sempre pre-allocate in memoria quando possibile, evitando rehashing e riallocazioni dispendiose. Questo significa che la maggior parte delle operazioni indicate con \complexityConstant{} ammortizzato siano in realtà totalmente costanti nella pratica. \\

\subsection{Misurazione del tempo di esecuzione}
\label{sub:stopwatch}

\noindent \codeinline{C++17} non fornisce soluzioni ad alto livello per registrare il tempo di esecuzione di singoli metodi o blocchi di codice. Abbiamo quindi implementato una funzione \codeinline{decorator} che astrae il compito di misurare i tempi di esecuzione di una funzione. \\

\noindent \codeinline{Shared/stopwatch\_decorator.h} definisce la
funzione generica \codeinline{decorator}, che:

\begin{itemize}
    \item riceve in input la funzione \codeinline{func} da eseguire e
      misurare;
    \item ritorna un'altra funzione che riceve in input i parametri
      variadici \codeinline{args} della funzione \codeinline{func};
    \item all'interno della funzione ritornata, viene avviato un
      cronometro;
    \item viene eseguita la funzione e ne viene catturato il risultato
      in una variabile contenitore;
    \item viene interrotto il cronometro e salvato il tempo di
      esecuzione;
    \item viene ritornato il risultato della funzione corredato del
      tempo di esecuzione. Ci sono due possibilità:
    \begin{enumerate}
        \item se il risultato della funzione è una tupla
          (\mintinline{c++}{std::tuple<...>}), il tempo di esecuzione
          è appeso in coda a tale tupla;
        \item se il risultato della funzione è invece un
          qualsiasi altro tipo, viene creata una nuova tupla
          contenente il risultato della funzione e il tempo di
          esecuzione.
    \end{enumerate}

    \noindent L'utilizzo di tuple facilita l'estrazione dei risultati nella funzione \codeinline{main()}, grazie agli \textit{structured bindings} offerti da C++17 \footnote{\url{http://www.open-std.org/jtc1/sc22/wg21/docs/papers/2015/p0144r0.pdf}}.
\end{itemize}

\noindent Si rimanda al codice allegato a questa relazione per il codice sorgente della funzione \codeinline{decorator}. Si veda invece il listing
\ref{listings:stopwatch-decorator-usage} per un'esempio di utilizzo di
tale funzione.

\begin{listing}[!ht]
\begin{minted}{c++}
auto graph = // ...
const size_t k = // ...
const auto program_time_start = // ...

// min_cut e discovery_time sono risultati della funzione karger
// karger_duration è il tempo di esecuzione della funzione in microsecondi
const auto [min_cut, discovery_time, karger_duration] =
    stopwatch::decorator<stopwatch::us_t>(karger)(graph, k, program_time_start);

\end{minted}
\caption{Esempio di utilizzo della funzione \codeinline{decorator} per
  rilevare i tempi di esecuzione di una funzione.}
\label{listings:stopwatch-decorator-usage}
\end{listing}

\subsection{Timeout per Karger}

\noindent L'algoritmo di Karger ha complessità temporale
\complexityKargerTime{}, quindi i tempi di esecuzione tendono ad
aumentare di molto in base al numero di nodi del grafo di input. Per
questo motivo abbiamo deciso, in maniera analoga a quanto fatto per
l'algoritmo di Held \& Karp dell'\emph{homework 2}, di creare un progetto
dove all'algoritmo di Karger abbiamo aggiunto un timeout di esecuzione
$T$. Abbiamo fissato il valore di $T$ a 2 minuti.\\

\noindent La nostra implementazione di Karger con Timeout, quindi:

\begin{enumerate}
    \item Ritorna il min\_cut migliore trovato in k esecuzioni di
      \codeinline{full\_contraction} se i suoi tempi di esecuzione
      sono inferiori a $T$ minuti, senza aspettare lo scadere del
      timeout;
    \item Se invece il timeout scade, termina preventivamente le k
      esecuzioni di \codeinline{full\_contraction} e ritorna il
      miglior min\_cut trovato fino a quel momento. A seconda del
      tempo impiegato da una singola esecuzione di
      \codeinline{full\_contraction}, l'algoritmo potrebbe ritornare
      il valore dopo qualche secondo dello scadere del timeout.
\end{enumerate}

\noindent C++17 non fornisce soluzione \textit{out-of-the-box} ad alto
livello per eseguire funzioni con un limite di tempo. Abbiamo quindi
implementato un meccanismo di questo tipo in
\codeinline{Shared/timeout.h}, il cui funzionamento ad alto livello è
il seguente:

\begin{itemize}
    \item Il thread principale crea un \textit{worker thread}
      incaricandolo di eseguire la funzione passata (in questo caso
      l'algoritmo di Karger) sul grafo letto in input. Fa quindi
      partire il timeout e resta in attesa del risultato del worker
      thread. Tale risultato sarà disponibile da un
      \codeinline{std::future}.
    \item Se il worker thread termina prima dello scadere del timeout,
      il thread principale è immediatamente sbloccato e il risultato
      della funzione (restituito da \codeinline{std::future::get()}) è
      ritornato al chiamante.
    \item Se il timeout scade e il worker thread non ha ancora
      terminato l'esecuzione, il thread principale gli notifica di
      terminare l'esecuzione il prima possibile. Tale notifica avviene
      forzando la conclusione di una \codeinline{std::promise} creata
      dal main thread e data in input alla funzione eseguita
      incapsulata nella classe \codeinline{timeout\_signal}.
    \item Quando la funzione eseguita si accorge che il tempo a
      disposizione è scaduto, interrompe la ricorsione e ritorna al
      chiamante la migliore soluzione individuata fino a quel momento.
\end{itemize}

\noindent Dato lo scarso valore aggiunto dal punto di vista del
codice, l'implementazione dell'algoritmo di Karger con il timeout non
è riportata sulla relazione; si rimanda invece ai sorgenti allegati
per la consultazione della stessa.
